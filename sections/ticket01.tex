% !TEX root = ../LinalColloc01.tex

\section{Множества и операции над ними}
\begin{conj} Множество - набор уникальных элементов \end{conj}

Множества - большие буквы $A, B,\dots$ \\
Элементы множеств - маленькие буквы $a, b,\dots$ \\
$x \in A - x$ пренадлежит $A$ \\
$x \notin A - x$ не пренадлежит $A$ \\
$\mathbb{N} = \{1, 2, 3, \dots\} \\
\mathbb{Z, Q} = \{{{m}\over{n}} : m \in \mathbb{Z}, n \in\mathbb{N}\} \\
\mathbb{R}$ - вещественные числа \\
$\mathbb{C}$ - комплексные числа \\
\begin{theorem-non} Правила Де Моргана \end{theorem-non}
    \begin{itemize}
        \item[] $A \; \setminus \; (\bigcup\limits_{\alpha \in I} B_{\alpha}) 
        = \bigcap\limits_{\alpha \in I}(A \setminus B_{\alpha})$

        \item[] $A \; \setminus \; (\bigcap\limits_{\alpha \in I} B_{\alpha}) 
        = \bigcup\limits_{\alpha \in I}(A \setminus B_{\alpha})$
    \end{itemize}
\begin{proof}
    Докажем для первой формулы. Вторая доказывается аналогично. \\
    $x \in A \; \setminus \; (\bigcup\limits_{\alpha \in I} B_{\alpha}) 
    \Longleftrightarrow \begin{cases}
        x \in A \\
        x \notin \bigcup\limits_{\alpha \in I} B_{\alpha}
    \end{cases}
    \Longleftrightarrow \begin{cases}
        x \in A \\
        x \notin B_{\alpha} \; \; $при всех$ \; \alpha
    \end{cases} 
    \Longleftrightarrow x \in A \; \setminus \; B_{\alpha}$ при всех $\alpha \in I
    \Longleftrightarrow x \in \bigcap\limits_{\alpha \in I}(A \setminus B_{\alpha})$ 
\end{proof}
\begin{theorem-non} Операции над множествами \end{theorem-non}
\begin{itemize}
    \item $A \cup B = \{x: x \in A $ или $ x \in B\}$
    \item $A \cap B = \{x: x \in A, x  \in B\}$
    \item $A \; \setminus \; B = \{x: x \in A, x  \notin B\}$
    \item $A \bigtriangleup B = (A \; \setminus \; B) \cup (B \; \setminus \; A)$
\end{itemize}
\notice - $\bigtriangleup, \cup, \cap$ - комммутативны, ассоциативны
\begin{conj} 
    Декартово произведение множеств 
    $A \times B = \{\langle a, b \rangle : a \in A; b \in B \}$ 
\end{conj}
\begin{theorem-non} \end{theorem-non}
    \begin{itemize}
        \item[] $A \cap \bigcup\limits_{\alpha \in I} B_{\alpha} =
        \bigcup\limits_{\alpha \in I}(A \cap B_{\alpha}) $

        \item[] $A \cup \bigcap\limits_{\alpha \in I} B_{\alpha} =
        \bigcap\limits_{\alpha \in I}(A \cup B_{\alpha}) $
    \end{itemize}
\begin{proof}
        $x \in A \cap \bigcup\limits_{\alpha \in I} B_{\alpha}
        \Longleftrightarrow \begin{cases}
            x \in A \\
            x \in \bigcup\limits_{\alpha \in I} B_{\alpha}
        \end{cases} \Longleftrightarrow \begin{cases}
            x \in A \\
            x \in B_{\alpha}$ для некоторых $\alpha \in I
        \end{cases}\vspace{0.5cm} \Longleftrightarrow 
        x \in A \cap B_{\alpha}$ для некоторых $\alpha \in I 
        \Longleftrightarrow 
        x \in \bigcup\limits_{\alpha \in I}(A \cap B_{\alpha})$
    \end{proof}
\begin{conj} 
    Упорядоченная пара $ \langle a, b \rangle $ - пара  ``пронумерованных'' элементов
\end{conj}
    $  \langle a, b \rangle $ = $  \langle c, d \rangle \rotatebox[origin=c]{150}{$\Longleftrightarrow$}$ 

\begin{scriptsize}
    \estiloJava
    \begin{lstlisting}[caption={}, label=]
        ((a == c) && (b == d))
    \end{lstlisting}
\end{scriptsize}