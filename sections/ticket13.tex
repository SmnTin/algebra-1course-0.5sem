\section{Корни из комплексных чисел}
\begin{normalsize}
\begin{theorem-non} Формула корня из комлпексного числа \end{theorem-non}
  Пусть $ z \in\mathbb{C},\  n \in\mathbb{N}$
  \begin{itemize}
    \item[1.] Если $z=0$, то уравнение $w^n=z$ имеет единственный корень 0.
    \item[2.] Если $z=r(\cos(\phi)+i\sin(\phi)), r>0$, то уравнение $w^n=z$ имеет ровно $n$ корней, а именно  $w_0, w_1, \dots, w_{n-1}$, где $w_k=\sqrt[n]{r}(\cos(\frac{\phi + 2\pi k}{n}) + i\sin(\frac{\phi + 2\pi k}{n})),\  k=0, 1, \dots, n-1$
  \end{itemize}

  \begin{proof}
    Первый случай тривиален. \\
    Пусть $w$ – число такое, что $w^n=z$. Очевидно, $z\neq 0$. \\
    $w:=\rho(\cos(\psi) + i\sin(\psi)), \ \rho > 0,\ \psi \in \mathbb{R}$ \\
    $w^n = z \Longrightarrow \begin{cases}
      \rho^n=r \\
      n\psi = \phi+2\pi k,\ k\in\mathbb{Z}
    \end{cases}
    \Longrightarrow \begin{cases}
      \rho=\sqrt[n]{r} \\ 
      \psi=\frac{\phi + 2\pi k}{n}\ ($для некоторого $ k\in \mathbb{Z})
    \end{cases}$ \\ 

    На этом этапе задача решена с одной оговоркой. Нужно понять, какие значения $\phi$ дают разные корни. \\

    $w_k := \sqrt[n]{r}(\cos(\frac{\phi + 2\pi k}{n}) + i\sin(\frac{\phi + 2\pi k}{n})) $ \\

    Среди этих $w_k$ могут встречаться одинаковые числа. Мы должны выяснить, какие из них будут одинаковыми. На самом деле предъявлены всех корни ($\{w_k\ |\ k\in \mathbb{Z}\}$), но мы даже не знаем мощность множества корней. Поймем, когда два таких корня равны: \\
    
    $w_k \stackrel{?}{=} w_l \Leftrightarrow \frac{\phi + 2\pi k}{n} = \frac{\phi + 2\pi l}{n} + 2\pi s, s\in\mathbb{Z} \Leftrightarrow \frac{k}{n} = \frac{l}{n} + s, s\in\mathbb{Z} \Leftrightarrow k-l \in n\mathbb{Z} (k \equiv l \mod n)$ \\ 

    То есть числа разные, если $k$ и $l$ имеют разный остаток при делении на $n$. Выходит, если мы хотим получить список всех (разных!) корней, то мы должны просто взять в качестве значений $k$ максимальный набор чисел, имеющих разный остаток при делении на $n$. \\

    Таким образом, множество чисел $\{ w | w^k = z\}$ совпадает со множеством $\{ w_k, k=0, \dots, n-1 \}.$
  \end{proof}
\end{normalsize}