\section{Группы. Подгруппы. Сокращение в группе}
\begin{conj}
    Группа - множество $G$ с заданной на нем бинарной операцией $\cdot$, такой что
    \begin{itemize}
    \item[] 1. $\cdot$ ассоциативна
    \item[] 2. $\cdot$ обладает нейтральным элементом
    \item[] 3. У любого элемента $G$ есть обратный
    \end{itemize}
\end{conj}


\begin{theorem-non}
$f: M \to N$ и у $f$ есть левое обратное ($g: N \to M, g \circ f = id_M$) $\Leftrightarrow f$ - инъективно

$f: M \to N$ и у $f$ есть правое обратное ($g: N \to M, f\circ g = id_M$) $\Leftrightarrow f$ - сюръективно
\end{theorem-non}

\begin{conj}
    $S_n$ - симметрическая группа степени $n$ (группа всех перестановок $1..n$)
\end{conj}
\begin{conj}
    Группа $(G, \cdot)$ называется абелевой, если $\cdot$ коммутативна
\end{conj}
\begin{itemize}
\item[] ассоциативность - полугруппа
\item[] полугруппа + нейтральный элемент - моноид
\item[] моноид + обратимость элементов - группа
\end{itemize}
\begin{conj}
    $(G, \cdot)$ - группа. $H \subset G$ называется подгруппой, если
    \begin{itemize}
    \item[] 1. $H\cdot H \subset H$
    \item[] 2. $e \in H$
    \item[] 3. $H^{-1} \subset H$
    \end{itemize}
\end{conj}