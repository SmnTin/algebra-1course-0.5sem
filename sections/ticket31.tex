\section{Формула Тейлора}

\begin{normalsize}

    \begin{theorem-non}
        $K$ --- поле, $f,g \in K[x], f,g \neq 0,\; d = deg(g) \geqslant 1.$ \\
        Тогда $f$ тождественным образом можно представить в виде:
        \begin{align}
            f = h_ng^n + h_{n-1}g^{n-1} + \dots + h_1g + h_0,
        \end{align}
        где $n \geqslant 0,\; h_n, \dots, h_0 \in K[x], h_n \neq 0$ и $deg(h_i) \leqslant d - 1$ при всех $i$
    \end{theorem-non}

    \textbf{О чём эта теорема?} \\
    \emph{Если у вас есть два ненулевых многочлена, степень $g$ хотя бы $1$. То $f$ можно разложить по степеням $g$ так,
    чтобы все коэфициенты имели степень не больше чем $deg(g) - 1$.
    Это нечто напоминающее разложение натурального числа, запись натурального числа в некоторой системе счисления. Мы там раскладываем
    его по степеням основания этой системы счисления, используя в качестве множителей цифры данной системы счисления.} \\
    \emph{Пример: \\
    Пятиричная система счисления, используем цифры от 0 до 4, то есть мы берём неотрицательные целые числа,
    которые строго меньше, чем основание системы счисления ($< 5$). \\
    }
    \emph{ Здесь делаем то же самое, в качестве множителей берём многочлены,
    степень которых строго меньше, чем $deg(g)$, чем степень того, на что делим.
    Последовательно делим с остатком. \\
    Делим  $f$ на $g$, остаток $h_0$.
    Неполное частное опять делим на $g$, остаток $h_1$, и так далее.
    Получаем нужное представление. \\
    (Ниже формальное доказательство)
    }

    \begin{proof}
        Пусть $l = deg(f)$. Напомним, что $d = deg(g)$. \\
        При $l < d$: \\
        Подойдёт $n = 0, h_0 = f$. Единственность очевидна из того, что по (1) $deg(f) = nd + deg(h_n)$, откуда $n=0$.
        
        При $l \geqslant d$:
        \begin{itemize}
            \item Сущесвтование:
                Имеем $f = gq + r$, где $deg(r) < d$. \\
                $deg(gq) = l$, откуда $deg(q) = l - d.$ \\
                Т.к. $0 \leqslant deg(q) < l$, по ИП существует представление:
                \begin{align*}
                    q = h_ng^{n} + h_{n-1}g^{n-1} + \dots + h_1g + h_0,
                \end{align*}
                откуда
                \begin{align*}
                    f = h_ng^{n+1} + h_{n-1}g^n + \dots + h_1g^2 + h_0g + r,
                \end{align*}
                и это равентсво - искомое представление $f$.

            \item Единственность:
                Заметим, что $n$ в (1) определено однозначно: из $deg(f) = nd + deg(h_n)$
                вытекает, что $n$ - неполное частное при делении $l$ на $d$.
                Далее, предположим, что вместе с (1) есть ещё одно представление:
                \begin{align*}
                    f = h'_ng^n + h'_{n-1}g^{n-1} + \dots + h'_1g + h'_0
                \end{align*}
                с аналогичными условиями на $h'_i$. \\

                Получаем: 
                \begin{align*}
                    g(h_ng^{n-1} + h_{n-1}g^{n-2} + \dots + h_1) + h_0 = 
                    g(h'_ng^{n-1} + h'_{n-1}g^{n-2} + \dots + h'_1) + h'_0
                \end{align*}
                и из условия единственности в теореме о делении с остатком получаем, что $h_0 = h'_0$, тогда:
                \begin{align*}
                    g(h_ng^{n-1} + h_{n-1}g^{n-2} + \dots + h_1) =
                    g(h'_ng^{n-1} + h'_{n-1}g^{n-2} + \dots + h'_1)
                \end{align*}
                откуда по идукционному предположению $h_n = h'_n, \dots, h_1 = h'_1$.
            \end{itemize}
    \end{proof}



\end{normalsize}