\section{Гомоморфизм подстановки}
\begin{normalsize}
    Для начала вспомним, что такое гомоморфизм кольца $R$ в кольцо $S$. \\
    Это такое отображение (не биективное) $\phi: R \rightarrow S$, что:
    \begin{enumerate}
        \item $\phi(a + b) = \phi(a) + \phi(b)$
        \item $\phi(a\cdot b) = \phi(a) \cdot \phi(b)$
        $\forall a,b \in R$
    \end{enumerate}
    Мы будем рассматривать только ассоциативные кольца с единицей и требовать
    от всех гомоморфизмов, чтобы они были унитарными, то есть выполнялось
    условие $\phi(1) = 1$.
    
    Примерами гомоморфизмов служат:
    \begin{enumerate}
        \item Любое вложение подкольца в кольцо (например,  $\Z$ в $\Q$)
        \item Отображение $\Z \rightarrow \Z /m\Z$, сопостовляющее целому числу
        его класс вычестов по модулю $m$
        \item Комплексное сопряжение
    \end{enumerate}

    \begin{theorem-non}
        Пусть $B$ --- кольцо, $A$ --- его подкольцо, такое что элементы
        $A$ коммутируют с любыми элементами $B\quad (\forall a \in A, b \in B: ab = ba)$. \\
        Зафиксируем $b \in B$. Тогда отображение
        \begin{align*}
            \phi_b: A[x] \rightarrow B
            a_nx^n + \dots + a_1x + a_0 \mapsto a_nb^n + \dots + a_1b + a_0
        \end{align*}
        представляет собой гомоморфизм колец.
    \end{theorem-non}

    \textbf{О чём речь?}

    \emph{Если у вас есть какое-то кольцо многочленов, то вы можете вместо переменных
    подставлять элементы самого этого кольца. И у вас получится такой гомоморфизм из
    кольца многочленов в само это кольцо. Если вы в сумму многочленов будете подставлять элементы,
    то получите тот же результат, как если бы сперва подставили в каждый многочлен элементы,
    а потом уже сложили. С произведением то же самое.} \\
    
    \textbf{Важно, что можно подставлять не только элементы самого кольца,
    но и некоторого его надкольца, не обязательно даже коммутативного.}


    При доказательстве теоремы $\phi_b(f)$ через $f(b)$; этот элемент можно назвать результатом подстановки $b$ в $f$ (вместо неизвестной).
    
    \begin{proof}
        Легко (да-да!) видеть, что $(f+g)(b) = f(b) + g(b)$, и что $1(b) = 1$.
        
        Осталось проверить, что $(fg)(b) = f(b)g(b)$. \\
        В силу дистрибутивности и уже доказанного свойства $(f + g)(b) = f(b) + g(b)$

        Достаточно рассмотреть случай мономов. Пусть $f = cx^m, g = cd^n$. Тогда
        
        \begin{align*}
            (fg)(b) = (cdx^{m+n})(b) = cdb^{m+n} = cdb^mb^n = cb^mdb^n = f(b)g(b)
        \end{align*}
    \end{proof}

    В частности, мы всегда можем подставить в многочлен $f \in R[x]$
    любой элемент самого кольца $x$. Результат подстановки тесно связан с делением с остатком, 
    как видно из следующей теоремы.






















\end{normalsize}
