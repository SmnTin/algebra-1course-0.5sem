% !TEX root = ../LinalColloc01.tex

\section{Существование и «единственность» аргумента комплексного числа}
\begin{enumerate}
    \item Если $z = 0$, то любое число является его аргументом.
    \item Если $z \neq 0$, то существует $\varphi_0 \in \mathbb{R}$ такое, что $\varphi_0$ - аргумент $z$, и при этом $\varphi_0 + 2\pi k$ для любого $k \in \mathbb{Z}$ тоже аргумент $z$.
    \begin{proof} 
        \begin{gather*}
            x^2 \leqslant x^2 + y^2 = r^2 \Rightarrow -r \leqslant x \leqslant r \\
            -1 \leqslant \frac{x}{r} \leqslant 1 \\
            \Rightarrow \exists\, \widetilde{\varphi} \in \mathbb{R} : \cos\widetilde{\varphi} = \frac{x}{r} \\
            (\sin\widetilde{\varphi})^2 = 1 - (cos\widetilde{\varphi})^2 = 1 - \frac{x^2}{r^2} = \frac{r^2 - x^2}{r^2} = \frac{y^2}{r^2} \\
            \sin\widetilde{\varphi} = \pm\frac{y}{r}
        \end{gather*}
        Если $\sin\widetilde{\varphi} = \frac{y}{r}$, то $\varphi_0 := \widetilde{\varphi}$, иначе $\varphi_0 := -\widetilde{\varphi}$.
  
        Итого получаем:
        \begin{gather*}
            \cos\varphi_0 = \frac{x}{r} \quad\quad \sin\varphi_0 = \frac{y}{r} \\
            \Rightarrow z = x + iy = r\cos\varphi_0 + r\sin\varphi_0i = r(\cos\varphi_0 + i\sin\varphi_0) \Rightarrow \varphi_0 - arg\,z
        \end{gather*}
        Покажем, что $\varphi = \varphi_0 + 2\pi k$ тоже является аргументом:
        \[ \varphi = \varphi_0 + 2\pi k \Leftrightarrow \cos\varphi = \cos\varphi_0,\, \sin\varphi = \sin\varphi_0 \Leftrightarrow \varphi - arg\,z \]
        Покажем, что все аргументы имеют вид $\varphi_0 + 2\pi k$. Для этого рассмотрим $\alpha \neq (\varphi_0 + 2\pi k)$. 
        Очевидно, что $\cos\alpha \neq \cos\varphi_0$ или $\sin\alpha \neq \sin\varphi_0$. Тогда $r * \cos\alpha \neq x$ или $r * \sin\alpha \neq y$. 
        Следовательно $r(\cos\alpha + i\sin\alpha) \neq z$.
    \end{proof}
\end{enumerate}
\underline{Свойства аргумента:}  
  
$z_1,\, z_2 \in \mathbb{C}^*$($\mathbb{C}$ без нуля)
\begin{enumerate}
    \item $arg\,z_1z_2 = arg\,z_1 + arg\,z_2$
    \begin{proof}
        \begin{gather*}
            z_1 \cdot z_2 = |z_1|(\cos\varphi_1 + i\sin\varphi_1) \cdot |z_2|(\cos\varphi_2 + i\sin\varphi_2) =  \\
            = |z_1||z_2|(\cos\varphi_1\cos\varphi_2 - \sin\varphi_1\sin\varphi_2 + i(\cos\varphi_1\sin\varphi_2 + \sin\varphi_1\cos\varphi_2)) = \\
            = |z_1||z_2|(\cos(\varphi_1 + \varphi_2) + i\sin(\varphi_1 + \varphi_2))
        \end{gather*}
    \end{proof}
    \item $arg\,\frac{z_1}{z_2} = arg\,z_1 - arg\,z_2$
    \begin{proof}  
        \[ z_1 = \frac{z_1}{z_2} \cdot z_2 \Rightarrow arg\,z_1 = arg\,\frac{z_1}{z_2} + arg\,z_2 \Rightarrow arg\,\frac{z_1}{z_2} = arg\,z_1 - arg\,z_2 \]
    \end{proof}
    \begin{notice}
        $arg\,\frac{1}{z} = arg\,1 - arg\,z = 0 - \arg\,z = -arg\,z$
    \end{notice}
    \item $arg\,\overline{z} = -arg\,z$
    \begin{proof}
        \begin{gather*}
            arg\,z = \varphi \\
            z = |z|(\cos\varphi + i\sin\varphi) \\
            \overline{z} = |z|(\cos\varphi - i\sin\varphi) = |\overline{z}|(\cos(-\varphi) + i\sin(-\varphi)) \\
            \Rightarrow arg\,\overline{z} = -arg\,z
        \end{gather*}
    \end{proof}
\end{enumerate}