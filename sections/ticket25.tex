% !TEX root = ../LinalColloc01.tex

\section{Факториальность евклидова кольца}
\begin{normalsize}
    Будем обозначать ассоциированные элементы как $a \  \mathtt{\sim} \  b$
    
    В области целостности $a \  \mathtt{\sim} \  b \Longleftrightarrow a = \varepsilon b$, где $\varepsilon \in R^*$ 
    (Один из другого получается путем умножения на обратимый)

    \begin{conj}
        Факториально кольцо - область целостности, в котором выполнены следующие свойства:
        \begin{enumerate}
            \item \textit{Любой элемент $a \in R$, отличный от нуля и не являющийся обратимым раскладывается в
            произведение неприводимых.} $a$ можно представить в виде $p_1 \dots p_s$, где $p_1 \dots p_s$ - неприводимые
            \item \textit{Такое разложение ``почти'' единственно (Можно подправлять порядок сомножителей и заменять их на ассоциированные).
            } Если оказалось, что $p_1 \dots p_s = q_1 \dots q_t$, где $p_1 \dots p_s$ и $q_1 \dots q_t$ 
            - неприводимые $(s, t > 0)$, то $s = t$ и после изменения порядка нумерации сомножителей, будет выполнено $p_1 \  \mathtt{\sim} \  q_1, \dots, p_s \  \mathtt{\sim} \  q_s$
        \end{enumerate}
    \end{conj}
    \textbf{ОЦ $\supset$ ФК $\supset$ ОГИ $\supset$ EO}
    \begin{theorem-non}
        Область главных идеалов - факториальное кольцо
    \end{theorem-non}
    \begin{proof}
        Пусть $R$ - евклидово кольцо, $\nu$ - соответствующая евклидова норма

        Проверим свойство 1 (существование разложения на неприводимые).
        Для этого докажем индукцией по $n$, что любой $a \neq 0$ с $\nu(a) \leqslant n$ либо обратим, 
        либо раскладывается в произведение неприводимых. 

        \underline{База индукции $n = 0$}: $a \in R^*$. Элмент с евклидовой нормой 0 обязательно обратим. Чтобы это показать 
        1 поделим с остатком на $a$. Если 1 не делится на $a$, то возникнет ненулевой остаток, а остаток должен
        иметь евклидову норму меньше, чем делитель. Так как меньше 0 она быть не может, то 1 делится на $a \Longrightarrow$ элемент обратим

        \underline{Переход}: Возьмем элемент с евклидовой нормой $= n$. Предположим он не обратим и не является неприводимым. Тогда она раскладывается 
        в произведение двух элементов $(a = bc)$, каждый из которых необратим. Тогда $\nu(b) < \nu(a)$ и $\nu(c) < \nu(a)$ по первому свойству евклидовой нормы. \\
        Тем самым, $b$ и $c$ раскладываются на неприводимые по индукционному предположению 

        Проверим свойство 2. НУО, предполагаем, что в равенстве $p_1 \dots p_s = q_1 \dots q_t$ выполнено $s \geqslant t$.
        Применим индукцию по $s$. Если $s = 1$, то выходит, что неприводимый разложился в произведение неприводимых. $t = 1$ и все доказано.
        Если $s > 1$ имеем $p_s \mid (q_1 \dots q_t)$, откуда $p_s \mid q_i$ при некотором $i$. Можно считать, что $i = t$, откуда $p_s \  \mathtt{\sim} \  q_t$,  то есть 
        $p_s \varepsilon q_t$, при некотором $\varepsilon \in R^*$. Получаем 
        \begin{gather*}
            (\varepsilon p_1)p_2 \dots p_{s - 1} = q_1 \dots q_{t - 1}
        \end{gather*}
        Остается только применить индукционное предположение
    \end{proof}
\end{normalsize}