% !TEX root = ../LinalColloc01.tex

\begin{normalsize}
\section{Кратные корни, сумма кратностей корней многочлена}

Вспомним, что по теореме Безу $f(a) = 0 \Leftrightarrow (x - a) \mid f$.

\begin{conj}
Пусть $f \in R[X]$, $f \neq 0$, $a \in R$ -- корень $f$. Его 
\textit{кратностью} называется наибольшее $n$, т.ч. $(X - a)^n \mid f$.
\end{conj}
\begin{conj}
Корень кратности 1 называется \textit{простым}, корень кратности 2
-- \textit{двойным}, кратности 3 -- \textit{тройным}... Корень кратности
$\geqslant$ 2 называется \textit{кратным корнем}.
\end{conj}
\begin{theorem-non}
Пусть $R$ -- поле, $f \in R[X]$, $d = \deg f \neq 0$; $a_1, \dots, a_s$
-- все его корни; $n_1, \dots, n_s$ -- их кратности. Тогда
\[n_1 + \dots + n_s \leqslant d\]
\end{theorem-non}
\begin{proof}
Т.к. многочлены степени 1 неприводимы, мы можем записать каноническое
разложение многочлена:
\begin{align*}
    & a = \varepsilon (X - a_1)^{n_1}(X - a_2)^{n_2}\dots(X - a_s)^{n_s}g
    \Longrightarrow \\
    &\Longrightarrow d = \deg (X - a_1)^{n_1} + \deg (X - a_2)^{n_2} +
    \dots + \deg (X - a_s)^{n_s} + \deg g = \\
    & = n_1 + n_2 + ... n_3 + \deg g \geqslant n_1 + n_2 + \dots + n_s
\end{align*}

$d$ в точности равно написанному выше выражению, т.к. поле является
областью целостности (обратимый элемент не может быть делителем нуля).
\end{proof}

\end{normalsize}