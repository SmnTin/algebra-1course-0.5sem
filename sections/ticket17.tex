\section{Теорема о делении с остатком для кольца целых чисел}
\begin{normalsize}

\begin{theorem-non}
    $f,g \in R[x],\; g \neq 0$ и старший коэфициент $g$ --- обратим. \\
    Тогда найдутся единственные многочлены $q,r \in R[x]$, такие что: \\
    \begin{enumerate}
        \item $f = g\cdot q + r$
        \item $deg(r) < deg(g)$
    \end{enumerate}
\end{theorem-non}

\begin{proof}
    Пусть $deg(g) = d,\; g = b_d\cdot x^d + \dots$.\\

    \textbf{Cуществование:}\\
    Докажем индукцией по $deg(f)$ \\
    Отметим, что при $deg(f) < d$ можно взять $q = 0,\; r = f.$ --- база. \\
    При $deg(f) = n \geqslant d$ обозначим через $a_n$ старший коэффициент $f$. \\
    Тогда у многочлена $a_n\cdot b_d^-1\cdot x^{n-d}\cdot g$ старший член равен $a_nx^n$ (как и у $f$), \\
    поэтому $deg(f-a_nb_d^{-1}x^{n-d}g) < n$ \\

    \textbf{Цитата к предыдущему шагу:}\\
    \emph{"Дальше мы вспоминаем как мы обычно делим многочлен на многочлен столбиком.
    Мы смотрим на старшие коэфициенты и делитель домножаем на $x$ в степени разность степеней
    и на отношение старших коэфициентов, в общем так, чтобы уровнять старшие коэфициенты.
    То есть мы $g$ домножаем на такой моном, чтобы у получившивося многочлена старшый член был такой же как у $f$.
    После того, как мы получившийся многочлен вычтем из $f$, мы получим какой-то многочлен меньшей степени."} \\
    
    По индукционному предположению найдутся некоторые многочлены $q_1,\; r_1$ такие, что: \\
    $deg(f-a_nb_d^{-1}x^{n-d}g) = gq_1 + r_1$ и $deg(r_1) < deg(g)$, отсюда: \\
    \begin{equation*}
        f = (a_nb_d^-1x^{n-d} + q_1)g + r_1
    \end{equation*}
    Это и есть искомое представление.

    \textbf{Единственность:} \\
    Пусть нашлось два таких представления: \\
    $f = gq_1 + r_1 = gq_2 + r_2 \\
    r_1 - r_2 = g(q_2 - q_1) \\
    deg(r_1 - r_2) < d = deg(g)$ (остатки изначально имели степень меньше $d$)\\
    $deg(g(q_2 - q_1)) = d + deg(q_2 - q_1)$ \\
    Пояснение: \emph{у $g$ старший коэфициент обратим, а у $f$ он не 0,
    поэтому при перемножении не может получиться 0}\\
    Если $q_2 - q_1 \neq 0$, то справа степень $\geqslant d$, а слева $< d$, значит $q_1 - q_2 = 0$, тогда $q_1 = q_2$ и $r_1 = r_2$
\end{proof}

\notice 
Наиболее важный случай когда $R$ --- поле.
При этом условии "старший коэфициент обратим" будет
выполняться автоматически для любого ненулевого многочлена $g$.

\notice
Как и в случае кольца целых чисел, $q$ и $r$ называются соответственно
неполным частным и остатком при делении $f$ на $g$.
Естественно, $f$ делится на $g \Leftrightarrow r = 0$

\end{normalsize}