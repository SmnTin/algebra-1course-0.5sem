\section{Свойства взаимно простых элементов в евклидовом кольце. Неприводимые элементы}
\begin{normalsize}
    Так как \textbf{области главных идеалов} $\supset$ \textbf{евклидовы кольца}, будем рассматривать все будем для ОГИ \\
    \begin{conj}
        Пусть $R$ - ОГИ 
        $a, b \in R$ называются взаимно простыми, если один из их наибольших общих делителей равен
    \end{conj}
    \textbf{Альтернативное определение:} $a$ и $b$ взаимно просты $\Longleftrightarrow \exists m,n \in R: am + bn = 1$ 
    \begin{proof} \quad \\
        ``$\Longrightarrow$'': из предыдущего предложения

        ``$\Longleftarrow$'': $\exists m,n \in R: am + bn = 1$. Пусть $d$ - НОД($a, b) \Longrightarrow d|a, \; d|b \Longrightarrow d|(am + bn) 
        \Longrightarrow d|1 \Longrightarrow d$ ассоциирован с 1 $\Longrightarrow 1$ - НОД($a, b$) 
    \end{proof}
    Если элементы $a$ и $b$ взаимно просты, то единицу можно представить в виде $am + bn$ и это условие равносильно условию взаимной простоты.

    Иными словами, взаимная простота равносильна существованию линейного представления единицы
    \begin{conj}
        Неприводимый элемент - ненулевой, необратимый элемент без нетривиального разложения (Оба множетеля не будут обратимыми)
    \end{conj}
    \textbf{Альтернативное определение:} Элемент $a \in R$ называется неприводимым, если $a \neq 0, \notin R^*$, и из 
    $a = bc$ следует, что либо $b \in R^*$, либо $c \in R^*$. Элементы $a \in R$ не являющиеся ни неприводимыми, ни обратимыми, ни 0, называют приводимыми
    
    \qquad Неприводимые элементы $\Z$ - простые числа. В кольце $K[X]$ неприводимыми будут многочлены степени 1. (Из $f = gh$  следует $deg \ f + deg \ g = 1$, откуда либо
    $deg \ f = 0$, либо $deg \ g = 0$.) Многочлены более высокой степени могут как быть, так и не быть неприводимыми: например, $X^2 + 1$ 
    неприводим как элемент $\R[X]$, но не как элемент $\mathbb{C}[X]$

    \textbf{Лемма:}
    \textit{Пусть $f \in K[X] \ - $ многочлен степени 2 или 3. Тогда он приводим, если и только если он имеет 
    корень в $K$}
    \begin{proof}
        Если $f(a) = 0$, то по теореме Безу $(X - a)|f$ и тем самым $f$ приводим.
        Обратно, если $f$ приводим, то $f = gh, deg \ g \geqslant 1$, и $deg \ h \geqslant 1$.
        Так как $deg \ g + deg \ h = deg \ f \leqslant 3$, то либо $deg \ g  = 1$, либо $deg \ h  = 1
        \Longrightarrow$ либо $h$, либо $g$ имеет корень в $K$, который также будет корнем $f$
    \end{proof}
    \qquad Однако многочлен степени $\geqslant 4$ может не иметь корней и быть при этом приводимым. Например, в кольце $\R[X]$ многочлен
    \begin{center}
        $X^4 + X^2 + 1 = (X^2 + 1)^2 - X^2 = (X^2 + X + 1)(X^2 - X - 1)$
    \end{center}
    приводим, но вещественных корней не имеет \\
    \textbf{Лемма:}
    \textit{Пусть $p, f \in R, p$ неприводимый. Тогда либо $p | f$, либо $(p, f) = 1$}
    \begin{proof}
        $(p, f)$ делит $p$ и поэтому либо обратим, либо ассоциирован с $p$. В последнем случае
        $p | f$
    \end{proof}
    \begin{theorem-non}
        Если $p$ неприводимый, и $p | ab$, то $p | a$ или $p | b$
    \end{theorem-non}
    \begin{proof}
        Предположим, $p \nmid a$ и $p \nmid b$. Тогда по лемме: $1 = pm + an$ и $1 = pm' + bn'$.
        Перемножим получим 
        \begin{gather*}
            1 = p(pmm' + mbn' + anm') + abnn',
        \end{gather*}
        откуда $p | 1$
    \end{proof}
\end{normalsize}