\section{Многочлены от одной переменной, переход к стандартной записи}
\begin{normalsize}
$R$ --- коммутативное ассоциативное кольцо с 1.

\begin{conj} 
    Многочленом от одной переменной над R называется финитная 
    последовательность $a_0, a_1,\dots$ элементов R, то есть 
    такая, что, $a_j \; \forall j \geqslant 0$ 
    при некотором $N \geqslant 0$ 
\end{conj}

$R[x]$ - множество всех многочленов от 1 переменной из $R$



\begin{theorem-non} Операции в R[x] \end{theorem-non}
    \begin{itemize}
        \item[] $+:$ $(a_0,a_1,\dots) + (b_0, b_1,\dots) = (a_0 + b_0, a_1 + b_1, \dots)$ --- финитная последовательность
        \item[] $\cdot:$ $(a_0,a_1,\dots) + (b_0, b_1,\dots) = (c_0, c+1, \dots)$, где $c_n = \sum_{k=0}^{n} a_k \cdot b_{n-k}$  
    \end{itemize}
\notice 
$(c_0, c+1, \dots)$ --- финитная последовательность.
Если $a_j = 0$ при $j \geqslant N_1,\;b_j$ при $N \geqslant N_2$,
то $c_j = 0$ при $j \geqslant N_1 + N_2$

Переход к стандартной записи.
\begin{itemize}
    \item[] $[a] := (a, 0, 0, \dots)$
    \item[] $[a] + [b] = [a+b]$
    \item[] $[a] \cdot [b] = [a\cdot b]$  
    \item[] $[a] \equiv a$
    \item[] $a \cdot (b_0, b_1, \dots) = (a\cdot b_0, a\cdot b_1, \dots)$
    \item[] $(a_0, a_1, \dots, a_n, 0, \dots) = \sum_{i=0}^{n} {(0,\dots,0,a_i,0,\dots)} = \sum_{i=0}^{n} {a_ix_i},\;$ где $x_i = (0,0,\dots,0,1,0,\dots)$ на $i$-том месте стоит 1.
    \item[] Получили вид: $a_0x_0 + a_1x_1 + \dots + a_nx_n$
    \item[] Заметим, что $x_k\cdot x_1 = x_{k+1}$ $\xRightarrow[\text{}]{\text{по индукции}}$ Каждое $x_k = x_1^k$
    \item[] Получили вид: $a_0 + a_1x_1 + a_2x_1^2 + \dots + a_nx_1^n$
    \item[] $x_1:=x$
    \item[] $a_0 + a_1x + \dots + a_nx^n$ 
\end{itemize}

\begin{theorem-non} $(R[x], +, \cdot)$ - коммутативное ассоциативное кольцо с 1 \end{theorem-non}
\begin{proof} 
    $ $
    \begin{enumerate}
        \item Четыре свойства сложения очевидны
        \item Коммутативность умножения очевидна
        \item Дистрибутивность следует из определения умножения: $c_n = \sum_{k=0}^{n} a_k \cdot b_{n-k}$, подставить $b_{n-k} = b'_{n-k} + b''_{n-k}$
        \item Ассоциативность: $(ab)c = a(bc)$ \\
        $a = \sum_{i=0}^{n} {a_ix^i},\;$ где $a_i \in R$ \\
        $b = \sum_{j=0}^{m} {a_jx^j},\;$ где $b_j \in R$ \\
        $c = \sum_{k=0}^{l} {a_kx^k},\;$ где $c_k \in R$ \\
        $(ab)c = 
        \sum_{i,j,k=0}^{n,m,l} {((a_ix^i)(b_jx^j))(c_kx^k)} =
        \sum_{i,j,k=0}^{n,m,l} {a_ib_jc_k((x^ix^j)(x^k))} =
        \sum_{i,j,k=0}^{n,m,l} {a_ib_jc_kx^{i+j+k}} = 
        a(bc)$
        \item Нейральный элемент по умножению --- 1.
    \end{enumerate}
\end{proof}
\end{normalsize}