\section{Формальное и функциональное равенство многочленов от одной переменной}
\begin{normalsize}
\begin{theorem-non}
    Пусть $R$ --- бесконечная область целостности, $f,g \in R[x]$ таковы,
    что они задают одинаковые отображения из $R$ в $R$, то есть
    $f(a) = g(a),\; \forall a \in R$
    Тогда $f = g$. 
\end{theorem-non}
Формальное - все коэфиценты равны.
Из формального следует функциональное равенство всегда.
\begin{proof}
    Любой элемент $a \in R$ служит корнем $f - g$.
    Но у ненулевого многочлена (над областью целостности) не может
    быть бесконечное множество корней. 
\end{proof}

\notice 
Тут важно, что $R$ --- бесконечная область. Например, если
$\mathbb{F}_p = \Z /p\Z$ --- поле из $p$ элементов ($p$ --- любое простое число), то
многочлены $x^p$ и $x$ принимают одинаковое значение при подстановке любого $a \in \mathbb{F}_p$ (это следует из малой теоремы Ферма)

Пара многочленов, формально различных, но функционально равных найдётся в любом кольце $R[x]$, где $R$ --- конечная область. (Докажите это!)
\end{normalsize}