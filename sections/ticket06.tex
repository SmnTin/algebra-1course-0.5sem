\section{Кольцо. Примеры. Группа обратимых элементов}
\begin{conj}
    Кольцо - множество $A$, на котором заданы две бинарные операции $+$ и $\cdot$ (называемые сложение и умножение), для которых выполняется следующее:
\end{conj}
\begin{enumerate}
    \item $(A, +)$ - абелева группа 
    \item $\forall a,\, b,\, c \in A \quad \begin{cases} a \cdot (b + c) = a \cdot b + a \cdot c \\  (b + c) \cdot a = b \cdot a + c \cdot a \end{cases}$ - дистрибутивность
\end{enumerate}
\begin{notice}
    Ассоциативность умножения также зачастаю относят к базовым свойствам, но мы выделяем это как дополнительное.
\end{notice}
  
$A$ называется
    \begin{itemize}
        \item ассоциативным кольцом, если умножение ассоциативно
        \item коммутативным кольцом, если умножение коммутативно
        \item кольцом с 1, если в $A$ существует нейтральный элемент относительно умножения
    \end{itemize}
  
\textbf{Примеры:}
\begin{enumerate}
    \item $(\mathbb{Z}, +, \cdot)$ - коммутативное, ассоциативное кольцо с 1 
    \item $(2\mathbb{Z}, +, \cdot)$ - коммутативное, ассоциативное кольцо без 1
    \item $(\mathbb{R}^3, +, \cdot$(вектрное умножение)) - не коммутативное, не ассоциативное кольцо без 1
    \item Пусть $M$ - множество. Тогда $(2^M, \otimes, \cap)$ - коммутативное, ассоциативное кольцо с 1 
    \begin{gather*}
        A \otimes B = B \otimes A \quad\quad (A \otimes B) \otimes C = A \otimes (B \otimes C) \\
        0 = \varnothing \\
        A \otimes A = \varnothing \\
        A  \cap (B \otimes C) = (A \cap B) \otimes (A \cap C) \\
        A \cap B = B \cap A \quad\quad A \cap (B \cap C) = (A \cap B) \cap C \\
        1 = M
    \end{gather*}
\end{enumerate}
\textbf{Лемма.} Пусть $A$ - кольцо, тогда $\forall a \in A \quad 0 \cdot a = 0$
\begin{proof}
    \begin{gather*}
        0 + 0 = 0 \\
        (0 + 0) \cdot a = 0 \cdot a \\
        0 \cdot a + 0 \cdot a = 0 \cdot a \\
        0 \cdot a + 0 \cdot a = 0 \cdot a + 0 \\
        0 \cdot a = 0
    \end{gather*}
\end{proof}
\begin{conj}
    Пусть $(A, +, \cdot)$ - кольцо. Тогда $B \subset A$ называется подкольцом $A$, если 
    \begin{itemize}
        \item $B$ замкнуто относительно умножения 
        \item $B$ является аддитивной подгруппой $A$
    \end{itemize}
\end{conj}
\vspace{0.7cm}
  
\begin{conj}
    Подмножество $I \subset A$ называется идеалом кольца $A$, если 
    \begin{itemize}
        \item $A$ - коммутативное, ассоциативное кольцо с 1 
        \item $I$ - аддитивная подгруппа $A$ 
        \item $\forall a \in A,\; \forall b \in I \quad a \cdot b \in I$
    \end{itemize}
    Например, $2\mathbb{Z}$ - идеал кольца  $\mathbb{Z}$.
\end{conj}
\begin{notice} 
  
    $\{ 0 \}$ всегда является идеалом в $A$ \\
    $A$ всегдя является идеалом в $A$
\end{notice}
\begin{conj}
    Пусть $A$ - коммутативное, ассоциативное кольцо с 1. Тогда идеал $I$ называется главным идеалом кольца $A$, если 
    \[ \exists\, a \in A : I = \{ a \cdot x \, | \, x \in A \} \] 
\end{conj}
\vspace{0.7cm}
  
\begin{conj}
    Пусть $A$ - ассоциативное кольцо с 1. Тогда $A^*$ - множество обратимых элементов $A$.
  
    Например, $\mathbb{Z}^* = \{ \pm1 \},\; \mathbb{Q}^* = \mathbb{Q}\, \backslash\, \{ 0 \}$
\end{conj}
\begin{theorem-non}
    $(A^*, \cdot)$ - группа обратимых элементов.
\end{theorem-non}
\begin{proof} \quad
    \begin{itemize}
        \item Ассоциативность есть, так как она есть в $A$ 
        \item 1 есть, так как $1 \cdot 1 = 1$
        \item Все элементы обратимы
        \item Замкнутость относительно умножения:
        \begin{gather*}
            a,\, b \in A^* \\
            \exists\, a' \in A^* : aa' = a'a = 1 \\
            \exists\, b' \in A^* : bb' = b'b = 1 \\
            (ab)(b'a') = a(bb')a' = aa' = 1 \\
            \Rightarrow (ba)' = b'a' \Rightarrow ab \in A^*   
        \end{gather*}
    \end{itemize}
\end{proof}