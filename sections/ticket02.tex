\section{Отображения. Инъективность. Сюръективность. Биективность}
Отображения, инъективность, сюръективность
\begin{conj} 
    $f: M \to N$ - отображение. Соответствие между      элементами двух множеств. $f(m)$ - образ под действием $m$
\end{conj}
\begin{conj}
    $f: M \to N$ инъективно, если $\forall m_1, m_2 \in M : m_1\neq m_2 \Rightarrow f(m_1)\neq f(m_2)$
\end{conj}
\begin{conj}
    $f: M \to N$ сюръективно, если $\forall n \in N\ \exists m \in M : f(m) = n$
\end{conj}
\begin{conj}
    $f: M \to N$ биективно, если оно инъективно и сюръективно
\end{conj}
\begin{conj}
    Пусть $n\in N, f: M\to N$. Тогда 
    
    $f^{-1}(n) = \{m \in M | f(m)=n\}$ - полный прообраз элемента
\end{conj}
\begin{itemize}
    \item[] $f$ инъективно $\Leftrightarrow \forall n \in N : |f^{-1}(n)|\leq 1$
    
    \item[] $f$ сюръективно $\Leftrightarrow \forall n \in N : |f^{-1}(n)| \geq 1$
    
    \item[] $f$ биективно $\Leftrightarrow \forall n \in N : |f^{-1}(n)| = 1$
\end{itemize}