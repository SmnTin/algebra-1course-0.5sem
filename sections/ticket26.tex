% !TEX root = ../LinalColloc01.tex

\begin{normalsize}
\section{$p$-ардический показатель и каноническое разложение}

\begin{conj}
    $p$-ардический показатель.
\end{conj}
    
Пусть $R$ -- любое факториальное кольцо, $p \in R$ --
неприводимый элемент в нём. Тогда для любого $a \in R$
определим $p$-ардический показатель
\[v_p(a) = \sup\{ n : p^n \mid a \}\]

\notice $v_p(0) = +\infty$

\begin{theorem-non}
Пусть $a \neq 0$. Тогда $v_p(a) = n$, если и только если $a = p^n c$,
$ p \nmid c $.
\end{theorem-non}
\begin{proof} $ $

``$\Longrightarrow$'':\\
Если $v_p(a) = n$, тогда $a = p^n c$.
Если бы $p \mid c$, тогда бы $v_p(a) > n$.

``$\Longleftarrow$'':\\
$a = p^n c \Rightarrow v_p(a) \geqslant n$. Пусть $v_p(a) \geqslant 
n + 1$. Тогда $a = p^{n + 1}d \Rightarrow c = pd \Rightarrow p \mid c$.
Противоречие.
\end{proof}

\begin{theorem-non}
Пусть $a \neq 0$, $b \neq 0$. Тогда $v_p(ab) = v_p(a) + v_p(b)$.
\end{theorem-non}
\begin{proof} $ $

\framebox[1.02\width]{В конспекте Жукова приводится куда более сложное
доказательство, не знаю зачем.}\\
\framebox[1.02\width]{Я написал своё, может, я ошибаюсь.} \par

Пусть $n := v_p(a)$, $m := v_p(b)$. Тогда $a = p^n c$, $b = p^m d$,
где $p \nmid c$, $p \nmid d$ по предыдущему предложению. 
Получаем, $ab = p^n c \cdot p^m d = p^{n + m} \cdot cd$. 
Если $p \mid cd$, значит $p \mid c$ или $p \mid d$
(т.к. $p$ по определению неприводим).
Но это неверно, значит $p \nmid cd$. Значит, $v_p(ab) = n + m$
по предыдущему предложению.

\end{proof}

Разложение элемента факториального кольца на неприводимые
неоднозначно в двух аспектах:
\begin{enumerate}
    \item можно менять порядок множителей;
    \item каждый множитель определён с точностью до ассоциированности.
\end{enumerate}
От второго можно избавиться, если в каждом классе ассоциированности
зафиксировать один элемент.

\textbf{Пример:}

Множество целых чисел (тоже факториальное кольцо). Любое целое число 
можно разложить на простые (в точности неприводимые в этом кольце).
Но при разложении можно менять знаки. Например, $6 = 2 \cdot 3 =
(-2) \cdot (-3)$. Можно условиться раскладывать только на положительные
простые, а если число было отрицательным, то выносить отдельно $-1$.
Например, $10 = 2 \cdot 5$, но $-10 = (-1) \cdot 3 \cdot 2$.
Тогда любое разложение будет единственным с точностью до перестановки.

\begin{theorem-non}
Пусть $P$ -- множество неприводимых элементов факториального кольца $R$,
содержащее ровно один элемент из каждого класса ассоциированных
неприводимых элементов. Тогда любой $a \in \mathbb{R}$, $a \neq 0$ 
представляется единственным образом с точностью до перестановки в виде 
финитного произведения
\[a = \varepsilon \prod_{p \in P} p^{v_p(a)}\]
где $\varepsilon \in R^*$.
\end{theorem-non}
\begin{proof}
Пусть $a = q_1 q_2 ... q_s$. Тогда $q_i = \varepsilon_i p_i$, где
$p_i \in P$ ассоциирован с $q_i$, а $\varepsilon_i \in R^*$. Тогда
$a = \varepsilon_1 p_1 \cdot \varepsilon_2 p_2 \cdot \dots \cdot 
\varepsilon_s q_s = (\varepsilon_1 \varepsilon_2 \dots \varepsilon_s) 
\cdot p_1 p_2 \dots p_s$ и $\varepsilon = \varepsilon_1 \varepsilon_2 
\dots \varepsilon_s$. Такое разложение единственно с точностью до
перестановки, т.к. в $P$ присутствует ровно один элемент из каждого
класса ассоциированности.
\end{proof}

\textbf{Пример:}

Как уже было сказано ранее, для $R = \mathbb{Z}$ за $P$ обычно берут
множество положительных простых чисел. Для $R = K[X]$, где $K$ -- поле,
обычно берут множество \textit{унитарных} (другими словами, приведённых, 
т.е. со старшим коэффициентом $1$) неприводимых многочленов. Таким
образом получается каноническое разложение многочлена:
\[f = \varepsilon p_1^{n_1} p_2^{n_2} \dots p_s^{n_s}\]
где $\varepsilon \in K$, $n_1, n_2, \dots, n_s \in \mathbb{N}$, а 
$p_1, p_2, \dots, p_s$ -- какие-то унитарные неприводимые многочлены,
причем $\varepsilon = 0$, если $a = 0$, иначе $\varepsilon \in K^*$.


\end{normalsize}