% !TEX root = ../LinalColloc01.tex

\begin{normalsize}
\section{Разложение многочлена по степеням заданного многочлена}

\begin{theorem-non}
    Пусть $K$ -- поле, $f, g \in K[X]$, $f \neq 0$,
    $d = \deg g \geqslant 1$. Тогда $f$ можно единственным
    образом представить в виде
    \[f = h_n g^n + h_{n-1}g^{n-1} + ... + h_1 g + h_0\]
    где $n \geqslant 0$, $h_0, \dots, h_n \in K[X]$,
    $h_n \neq 0$ и $\forall i \,\, \deg h_i \leqslant 
    d - 1$.
\end{theorem-non}

\textbf{О чём эта теорема?} \\
    \emph{Если у вас есть два ненулевых многочлена, степень $g$ хотя бы $1$. То $f$ можно разложить по степеням $g$ так,
    чтобы все коэфициенты имели степень меньше чем $deg(g)$.
    Это нечто напоминающее разложение натурального числа, запись натурального числа в некоторой системе счисления. Мы там раскладываем
    его по степеням основания этой системы счисления, используя в качестве множителей цифры данной системы счисления.} \\
    \emph{Например, $10 = 1 \cdot 3^2 + 0 \cdot 3 + 1 = 101_3$.} \\
    \emph{ Для многочленов делаем то же самое, в качестве множителей берём многочлены,
    степень которых строго меньше, чем $deg(g)$ (степень того, на что делим).
    Последовательно делим с остатком. \\
    Делим  $f$ на $g$, остаток $h_0$.
    Неполное частное опять делим на $g$, остаток $h_1$, и так далее.
    Получаем нужное представление. \\
    (Ниже формальное доказательство)
    }

\begin{proof}
    Используем индукцию по $l = \deg f$.

    При $l < d$ подходит $n = 0$, $h_0 = f$. Из записи доказываемого
    разложения видно, что $\deg f = nd + \deg h_n$ (что верно, т.к.
    поле -- область целостности), откуда $n = 0$, из чего следует, что
    это разложение единственное.

    При $l \geqslant d$ имеем $f = gq + r$, где $g, r \in K[X]$,
    $\deg r < d$ (такое разложение существует, т.к. $g \neq 0$ и
    старший коэффициент $g$ обратим). Т.к. $\deg r < d$, получается,
    что $l = \deg(gq)$, откуда $\deg q = l - d$ (всё это верно, т.к.
    $K$ -- область целостности). Т.к. $0 \leqslant q < l$ по 
    индукционному предположению существует следующее разложение
    \[q = h_n g^n + h_{n-1} g^{n-1} + \dots + h_1 g + h_0\]
    Тогда
    \[f = gq + r = h_n g^{n+1} + h_{n-1} g^n + \dots + h_1 g^2 + 
    h_0 g + r\]
    Приведённое выше равенство и есть искомое разложение, причём
    по теореме о делении с остатком такие $q$ и $r$ для заданных
    $f$ и $g$ единственны. По индукционному предположению разложение
    $q$ тоже единственно. Тогда и разложение $f$ также единственно.

    \framebox[1.02\width]{В конспекте Жукова приводится чуть более 
    сложное доказательство единственности.}\\
    \framebox[1.02\width]{Я переписал, может, я ошибаюсь.} \par

\end{proof}

\end{normalsize}