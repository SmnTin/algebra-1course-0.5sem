\section{Теорема Безу. Число корней многочлена над областью}
\begin{normalsize}
\begin{theorem-non}
    Пусть $f \in R[x],\; c \in R$. Тогда остаток при делении $f$ на $(x - c)$ равен $f(c)$.
\end{theorem-non}

\begin{proof}
    $ $ \\
    \begin{center}
        $f = (x-c)g + r$\\
        $f(c) = (c - c)g + r(c) = r$\\
        $r$ --- константа, так как мы делим на многочлен степени 1.
    \end{center}
    
\end{proof}

\notice Иногда это утверждение называют второй теоремой Безу, ну а первой теоремой Безу оказывается частный случай: $f(c) = 0 \leftrightarrow f$ делится на $(x-c)$.

Пусть $f \in R[x]$. Элементы кольца $R$ (или какого-то его надкольца S) такие, что $f(c) = 0$ называются корнями $f$ в $R$ (соответственно в S).
Например, многочлен $x^4 - 2 \in \Z[x]$ имеет $0$ корней в $\Z$, $2$ корня в $\R$, $4$ корня в $\mathbb{C}$.

\begin{theorem-non}
    Пусть $R$ - область целостности, $f \in R[x],\; deg(f) = d \geqslant 0$. \\
    Тогда число корней $f$ в $R$ не превосходит $d$.
\end{theorem-non}

\begin{proof}
    Используем индукцию по $d$. \\
    При $d = 0$ утверждение очевидно (ненулевая константа не имеет корней). \\
    Пусть $d > 0$. Если у $f$ нет корней в $R$, то утверждение теоремы выполнено. \\
    В противном случае пусть $c_1, \dots, c_l$ --- все корни $f$ в $R$. Тогда по теореме Безу
    $f = (x - c_l)g$ для некоторого $g \in R[x]$ \\
    Для каждого $i$ от 1 до $(l-1)$ имеем $f(c_i) = (c_i - c_l)g(c_i)$\\
    При этом $f(c_i) = 0$, но $c_i - c_l \neq 0$. Воспользовавшись тем, что $R$ --- область целостности, мы заключаем, что $g(c_i) = 0$
    Таким образом, $c_1,\dots,c_{l-1}$ --- корни $g$, при этом ясно, что $deg(g) = d - 1$
    По индукционному предположению получаем $l - 1 \leqslant d - 1$, откуда $l \leqslant d$.
\end{proof}

\notice
Тут важно, что $R$  --- область целостности. Например, многочлен $x^2 - \bar{1}$, рассматриваемый как многочлен над $\Z / 8\Z$ (или $\Z / 15\Z$) имеет четыре корня. (Проверьте это!!!!!!!!!)
\end{normalsize}