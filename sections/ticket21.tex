\section{Делимость и ассоциированные элементы. Определение НОД}
\begin{normalsize}
    $R$ обозначает произвольное евклидово кольцо, $\nu$ - соответствующую евклидову норму
    \begin{theorem-non}
        Пусть есть $a, b \in R$
        \begin{conj}
            Элемент $x$ делит элемент $y \Longleftrightarrow \exists z: \; y = x \cdot z$ ($y$ лежит в главном идеале, порожденным элементом $x$)
        \end{conj}
        \begin{conj}
            Элемент $d \in R$ называется наибольшим общим делителем $a$ и $b$, если \begin{itemize}
                \item $d | a, d | b$
                \item Если $\exists d' : d' | a, d' | b$, то $d' | d$
            \end{itemize}
        \end{conj}
    \end{theorem-non}
        \begin{conj}
            Элементы $a, b \in R$ назыываются ассоциированными, если $a|b$ и $b|a$
        \end{conj}
        \textbf{Лемма.}
        \textit{Пусть $d$ - НОД $a$ и $b$, и $d'$ - элемент $R$. \\
        Тогда $d'$ тоже НОД $a$ и $b \Longleftrightarrow d'$ ассоциирован с $d$}
\end{normalsize}