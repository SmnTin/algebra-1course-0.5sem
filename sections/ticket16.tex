% !TEX root = ../LinalColloc01.tex

\section{Свойства степени многочлена}
\begin{normalsize}
\begin{conj}
Степенью многочлена $f$ называется максимальное $n$, для которого
$a_n \neq 0$. Обозначается как $\deg f$. \\
При этом $a_n$ --- старший коэффициент, 
а $a_nx^n$ --- старший член.
\end{conj}
\notice У нулевого многочлена степень равна $-\infty$ или $-1$ (соглашение).
\begin{theorem-non}
    Пусть $f,g \in R[x],\; \deg f = m,\; \deg g = n.$
    \begin{enumerate}
        \item $\deg(f+g) \leqslant \max\{m,n\}$ --- говорит о том,
        что у противоположных по знакам старших коэфициентов
        старший член может стать равен 0 и степень будет меньше.
        Равенство достигается при $m \neq n$
        \item $\deg(f\cdot g) \leqslant m + n$
    \end{enumerate}
\end{theorem-non}

\begin{proof}
    Если один из многочленов равен нулю,
    утверждения очевидны ($- \infty + smth = -\infty$).
    Далее $n,m \geqslant 0$
    \begin{enumerate}
        \item Пусть $d = \max \{n,m\}$, тогда 
        $f$ и $g$ представим в виде $\sum_{i=0}^{d} {a_ix^i}$ и $\sum_{i=0}^{d} {b_ix^i}$.
        \emph{Если у нас степень одного из многочленов меньше чем $d$, то мы можем дописать к нему в конце нулевые слагаемые, чтобы заканчивалось всё на $x^d$.} \\
        $f + g = \sum_{i=0}^{d} {(a_i + b_i)x^i}$ и $deg(f+g) \leqslant d$ \\
        Если $m\neq n$, то $a_d + b_d \neq 0$ (сложение нулевого и ненулевого элемента), тем самым $deg(f+g) = d$.
        \item $f = \sum_{i=0}^{n} {a_i\cdot x^i},\; g = \sum_{j=0}^{m} {b_j\cdot x^j}$ \\
        $f\cdot g = \sum_{i,j=0}^{n,m} {a_i\cdot b_j \cdot x^{i+j}}$ и $deg(f\cdot g) \leqslant m + n.$
    \end{enumerate}
\end{proof}

\notice
$\deg(f\cdot g)$ может оказаться меньше, чем $\deg f + \deg g$, если в кольце $R$ есть делители нуля $(a_n\cdot b_m = 0)$. \\
Это значит, что $R$ -- это не область целостности. \\
Пример: $f = \bar{2}x + \bar{1}$ в кольце $\Z / 4\Z[x]$ \\
$\deg f = 1$ \\
$\deg(f\cdot f) = 0$, так как $(\bar{2}x + \bar{1})^2=\bar{1}$ \\
$\deg f + \deg f= 1 + 1 = 2$ \\

\begin{theorem-non}
    Пусть $R$ - область целостности.
    \begin{enumerate}
        \item $\forall f,g \in R[x]: \deg(f\cdot g) = \deg f + \deg g$
        \item $R[x]$ --- область целостности.
    \end{enumerate}
\end{theorem-non}

\begin{proof}
    $ $
    \begin{enumerate}
        \item Пусть $m = \deg f,\; n = \deg g,\; a_m,\; b_n$ - соответствующие старшие коэфициенты. \\
        Тогда в многочлене $fg$ коэфициент $a_m\cdot b_n \neq 0$ при $x^{m+n} \Rightarrow \deg(f\cdot g) = m + n = \deg f + \deg g$    
        \item Следует из первого: $f,g \neq 0 \Rightarrow \deg(f\cdot g) = \deg f + \deg g \geqslant 0 + 0$
    \end{enumerate}
\end{proof}
\follow \; Пусть $R$ -- область целостности. Тогда $R[x]^* = R^*$
\begin{proof}
    $ $
    \begin{itemize}
        \item[$\subset$:] $R^* \subset R[x]^*$
        \item[$\supset$:] $f \in R[x]^*$ \\
        Тогда $f \neq 0$ и $f\cdot f^{-1} = 1$ \\
        $\deg 1 = \deg(ff^{-1}) = \deg f + \deg f^{-1} \geqslant \deg f$ $\Rightarrow \deg f = \deg 1 = 0$ \\
        Получается, что $f \in R$, ну а тогда $f \in R^*$ \\

        \textbf{Цитата из лекции, чтобы было проще понимать это доказательство:} \\
        \emph{"Если же теперь у вас есть обратимый многочлен, то у него \textbf{не} может быть степень больше нуля,
        потому что когда вы будете на что-то его домножать, ненулевое, то степень при этом будет лишь
        увеличиваться. Она будет оставаться такой же или увеличиваться и единицу вы не получите.
        Значит остаётся рассматривать константы, а константы надо брать обратимые}
        (чтобы получить единицу - прим.)."
        \item[Пример:] $\Z[x]^* = \{ \pm 1 \}$. В то же время $(\Z / 4\Z [x])^*$ --- бесконечное множество.
    \end{itemize} 
\end{proof}

\end{normalsize}